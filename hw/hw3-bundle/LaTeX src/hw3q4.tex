\section{Dense Communities in Networks (20 points)}

In this problem, we study the problem of finding dense communities in networks. 

\paragraph{Definitions:}
Assume $G=(V,E)$ is an undirected graph (e.g., representing a social network).
\begin{itemize}
  \item For any subset $S\subseteq V$, we let the \textit{induced edge set} (denoted by $E[S]$) to be the set of edges both of whose endpoints belong to $S$.
  \item For any $v\in S$, we let $\mbox{deg}_{S}(v)= |\{u\in S| (u,v)\in E\}|$.
  \item Then, we define the \textit{density} of $S$ to be:
	\[
		\rho(S) = \frac{|E[S]|}{|S|}.
	\]
  \item Finally, the \textit{maximum density} of the graph $G$ is the density of the densest induced subgraph of $G$, defined as:
	\[
		\rho^{*}(G) = \max_{S\subseteq V} \{\rho(S)\}.
	\]
\end{itemize}

\paragraph{Goal.} Our goal is to find an induced subgraph of $G$ whose density is not much smaller than $\rho^{*}(G)$. Such a set is very densely connected, and hence may indicate a community in the network represented by $G$. Also, since the graphs of interest are usually very large in practice, we would like the algorithm to be highly scalable. We consider the following algorithm:

\begin{algorithmic}
\REQUIRE $G = (V, E)$ and $\epsilon > 0$
\STATE $\tilde{S}, S\leftarrow V$
\WHILE {$S\neq\emptyset$}
\STATE $A(S):=\left\{ i\in S \mid \deg_S(i) \leq 2(1+\epsilon)\rho (S) \right\} $
\STATE $S\leftarrow S \setminus A(S)$
\IF {$\rho(S)>\rho(\tilde{S})$}
\STATE $\tilde{S} \leftarrow S$
\ENDIF
\ENDWHILE
\RETURN $\tilde{S}$
\end{algorithmic}

The basic idea in the algorithm is that the nodes with low degrees do not contribute much to the density of a dense subgraph, hence they can be removed without significantly influencing the density.

We analyze the quality and performance of this algorithm. We start with analyzing its performance.

\subquestion{(a) [10 points]} We show through the following steps that the algorithm terminates in a logarithmic number of steps.
\begin{enumerate}[i.]
  \item Prove that at any iteration of the algorithm, $|A(S)|\geq \frac{\epsilon}{1+\epsilon} |S|$.
  \item Prove that the algorithm terminates in $O(\log_{1+\epsilon}{(n)})$ iterations, where $n$ is the initial number of nodes.
\end{enumerate}



\subquestion{(b) [10 points]} We show through the following steps that the density of the set returned by the algorithm is at most a factor $2(1+\epsilon)$ smaller than $\rho^{*}(G)$.
\begin{enumerate}[i.]
  \item Assume $S^{*}$ is the densest subgraph of $G$. Prove that for any $v\in S^{*}$, we have: $\mbox{deg}_{S^{*}}(v) \geq \rho^{*}(G)$.
  \item Consider the first iteration of the while loop in which there exists a node $v\in S^{*}\cap A(S)$. Prove that $2(1+\epsilon)\rho(S)\geq \rho^{*}(G)$.
  \item Conclude that $\rho(\tilde{S}) \geq \frac{1}{2(1+\epsilon)}\rho^{*}(G)$.
\end{enumerate}



\subsection*{What to submit}
\begin{enumerate}[(a)]
    \item
        \begin{enumerate}[i.]
            \item Proof of  $|A(S)|\geq \frac{\epsilon}{1+\epsilon} |S|$.
            \item Proof of  number of iterations for algorithm to terminate.
        \end{enumerate}
    \item
        \begin{enumerate}[i.]
            \item Proof of  $\mbox{deg}_{S^{*}}(v) \geq \rho^{*}(G)$.
            \item Proof of $2(1+\epsilon)\rho(S)\geq \rho^{*}(G)$.
            \item Conclude that $\rho(\tilde{S}) \geq \frac{1}{2(1+\epsilon)}\rho^{*}(G)$.
        \end{enumerate} 
\end{enumerate}


